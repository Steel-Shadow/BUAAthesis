% !Mode:: "TeX:UTF-8"

% 中英文摘要
\begin{cabstract}
    基于 Transformer 架构的大语言模型(Large Language Model, LLM)使自然语言处理领域取得了巨大进步。然而,由于这些模型的推理延迟较长,它们在各种实时应用中的使用受限。自回归生成任务进一步加剧了推理延迟,因为模型需要迭代运行,按顺序生成标记,不能利用标记级并行化。为了解决这个问题,该文提出了Big Little Decoder(BiLD)框架,它可以在不降低生成质量的前提下提高推理效率,适用于各种文本生成应用。BiLD 框架包含两个不同大小的模型,它们协同生成文本。小模型以自回归方式运行,以较低的推理成本生成文本;大模型偶尔以非自回归方式修正小模型的不准确预测。为了协调大小模型,BiLD 引入了两种简单而有效的策略:(1) Fallback 策略用于确定何时将控制权移交给大模型;(2) Rollback 策略用于确定大模型如何修正小模型不准确的预测。在实验评估部分,BiLD 应用于各种文本生成场景,包括 IWSLT 2017 De-En 和 WMT 2014 De-En 上的机器翻译,以及 XSUM 和 CNN/DailyMail 上的摘要生成。在 NVIDIA T4 GPU 上,在几乎不减低生成质量的前提下,BiLD 实现了高达 2.12 倍的速度提升。此外,BiLD 框架完全即插即用,无需修改训练过程或模型架构。
\end{cabstract}

% \begin{eabstract}
% Here is the Abstract in English. And this is a test sentence, just for
% a test to see how the buaathesis works. You can just ignore this.\par
% This is another pargraph.
% \end{eabstract}